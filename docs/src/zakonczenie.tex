\chapter{Zakończenie i wnioski}
\section{Podsumowanie}
%%%%%%%%%%%%%%%%%%%%%%%%%%%%%%%%%%%%%%%%%%%%%%%%%%%%%%%%%%%%%%%%%%%%%%%%%%%%%%%%
%%%%  DOPISAĆ O TYM CO UDAŁO SIĘ ZROBIĆ -- DODANO NEGACJE DO ANALIZY SENTYMENTU
%%%%%%%%%%%%%%%%%%%%%%%%%%%%%%%%%%%%%%%%%%%%%%%%%%%%%%%%%%%%%%%%%%%%%%%%%%%%%%%%

W niniejszej pracy próbowano powiązać ze sobą analizę sentymentu i dane
geolokacyjne w celu wzbogacenia analizy użytkowników sieci społecznościowych.
Cel można uznać za osiągnięty.

Analiza sieci społecznych pozwala pokazać w jaki sposób osoby badane łączą się
ze sobą i jaka jest charakterystyka tych połączeń. Pokazuje wielkość
utworzonych grup i ich trwałość.

Zastosowanie analizy sentymentu dostarcza informacje, dzięki którym
możemy odkrywać przyczyny tworzenia się takich a nie innych grup
a także pomaga dowiedzieć się, dlaczego między różnymi grupami występują
konkretne rodzaje relacji. Korzystając z tej dziedziny nauki możemy także
skorelować nastroje społeczne z wydarzeniami na świecie. W niektórych przypadkach
zauważony sentyment może być w miarę przewidywalny -- tak jak w badaniu 
kibiców piłkarskich, gdzie związany jest z wynikiem meczu -- ale w innych
może pozwalać odkrywać niezauważane do tej pory ciągi przyczynowo-skutkowe.

Analiza sieci społecznych może również prowadzić do bogatszych wniosków
poprzez zastosowanie danych związanych z geolokalizacją. Powiązanie informacji
o fizycznym położeniu użytkowników, a także o sentymencie jaki generują
daje dodatkowy obraz pozwalający zrozumieć połączenia między nimi.

Dodatkowo w tej pracy udało się wykorzystać dane pochodzące z serwisu 
społecznościowego -- którym był Twitter.
Jest to bardzo ciekawe medium, które może być szeroko używane do automatycznego
badania nastrojów i sieci społecznych. Pokazano w jaki sposób takie dane
uzyskać, przetworzyć i wykorzystać do podobnych badań. Zaprezentowane zostało podejście,
dzięki któremu badanie dużych grup ludzi można przeprowadzić bez ich wiedzy
i w sposób automatyczny. Dzięki temu uzyskuje się bardziej wiarygodne wyniki.
Użytkownicy nie wiedzą, że są obserwowani i zachowują się w sposób naturalny.


\section{Wpływ pracy na otaczający świat}
Praca ta prezentuje w jaki sposób można połączyć ze sobą analizę sieci społecznych,
analizę sentymentu i analizę geolokalizacji. Udowadnia, że badanie nastrojów
w społeczeństwie można zautomatyzować aplikując komputerowe techniki przetwarzania
dużych zbiorów danych. Pokazuje, że zastosowanie analizy sentymentu i geolokacji
może istotnie wzbogacić analizę dużych grup ludzi.

Zastosowanie automatycznych technik badania dużych sieci społecznych
może uprościć sposoby komunikacji z takimi grupami i odpowiadania na ich potrzeby.
Możemy wyobrazić sobie sytuację, w której rządy, organizacje czy firmy badając
sieci społeczne wraz z analizą i geolokacją błyskawicznie reagują na aktualne
wydarzenia. Przykładem zastosowania takich badań może być prezentowanie
spersonalizowanych reklam. Gdy na przykład dana drużyna przegrywa, jej kibicom
można by wyświetlać po zakończonym meczu inny zestaw reklam niż kibicom drużyny 
przeciwnej. Firmy oferujące swoje usługi czy produkty na całym świecie
mogą szybko reagować na opinie czy błędy zgłaszane przez sfrustrowanych internautów
w serwisach społecznościowych poprawiając swój wizerunek i pokazując dbałość
o klienta. Rządy czy partie polityczne mogą wykorzystać sieci społeczne,
sentyment i geolokacje do odpowiednich zmian, ustaw dotyczących konkretnych grup
społecznych, aby polepszyć wśród nich swoje notowania.

Praca pokazuje, że wykorzystanie mediów internetowych, czy
serwisów społecznościowych może dać wymierne korzyści. Potwierdza, że
świat wirtualny i realny przenikają się. Tak samo reagujemy na różne
wydarzenia niezależnie od tego, czy dzielimy się swoimi opiniami z bliskimi będącymi obok
nas czy z całym światem korzystając z serwisów społecznościowych. Jeśli więc
zarządy firm lub organizacji nie wierzą lub wahają się nad sensem przeprowadzenia
takich badań, to niniejsza praca może być dla nich dowodem, że warto bliżej
przyjrzeć się zaprezentowanym tutaj aspektom. Może być więc punktem wyjścia
do prowadzenia własnych badań na interesujące dany podmiot tematy.



\section{Możliwe kierunki rozwoju}
% plan dalszego rozwoju systemu
% w pływ pracy, zastosowanie wyników teraz i w przyszłości

Zaprezentowana praca jest ukierunkowana na wąską dziedzinę -- bada zachowania
kibiców piłkarskich. W związku z tym, aby rozszerzyć jej zakres konieczne
jest przeprowadzenie kilku działań celem jej rozwoju. Kilka możliwych kierunków to:

\subsubsection{Rozszerzenie o inne języki naturalne}
Zastosowany mechanizm analizy sentymentu jest skupiony jedynie na
języku angielskim. Aby móc badać większe grupy ludzi koniecznym jest opracowanie
techniki badającej również inne języki. Oczywiście w pewnym stopniu można 
wykorzystać podejście zastosowane w tej pracy, należy jednak mieć na uwadze
różnice między budową używanych języków. Zupełnie inne konstrukcje językowe
są w języku angielskim a zupełnie inne na przykład w języku polskim.
Oczywiste jest więc, że nie można w taki sam sposób podejść do badania
wpisów w różnych językach. Rozbudowanie mechanizmu o kolejne języki pozwoliłoby
uzyskać bogatsze wyniki.

\subsubsection{Odkrywanie tematów rozmów}
Ciekawym rozszerzeniem badania sieci społecznych w kontekście przetwarzania
języka naturalnego byłoby opracowanie i zastosowanie metody pozwalającej
odkrywać tematy rozmów użytkowników. W zaprezentowanym podejściu badany jest
jedynie sentyment wpisów, nie ma natomiast informacji na temat tego o czym dokładnie
dyskutują użytkownicy Twittera. Użycie mechanizmu ekstrakcji tematów rozmów
z wpisów z pewnością wzbogaciłoby analizę sieci społecznych.


\subsubsection{Wzbogacenie techniki badania sentymentu}
Badanie wydźwięku wypowiedzi zostało zautomatyzowane. Wykorzystuje do tego celu
wpisy tworzone przez badaną grupę. Interesującym rozszerzeniem techniki badania 
sentymentu mogłoby być zastosowanie ręcznie stworzonego słownika.
Mógłby on nieco skorygować niektóre wyniki analizy wydźwięku wypowiedzi i
polepszyć ich jakość. Dodatkowo pozwoliłoby to na zastosowanie metody analizy sentymentu
w szerszej dziedzinie badań. Aktualna metoda jest w pewien sposób zorientowana
na środowisko piłkarskie i niekoniecznie dawałaby dobre rezultaty w innym
zbiorze danych -- na przykład wpisach politycznych. Skorzystanie ze słownika
i zwięszkenie zakresu badanych wpisów pozwoliłoby szerzej zastosować
opracowaną metodę.

\subsubsection{Bliższe przyjrzenie się grupom}
Dodatkowym kierunkiem rozwoju mogło by być bliższe przyjrzenie się grupom,
które się utworzyły. Można by na przykład skierować swoje badania
na szukanie liderów tych grup, przyczyn ich pojawiania się i sposobów w jaki
oddziałują na resztę grupy w zależności od sentymentu jaki sami generują,
jaki generują dane grupy czy w jakiej lokalizacji geograficznej się znajdują. 




