\begin{abstract}

Był na Żmudzi[1] ród możny Billewiczów, od Mendoga[2] się wywodzący, wielce
skoligacony[3] i w całym Rosieńskiem[4] nad wszystkie inne szanowany. Do urzędów
wielkich nigdy Billewiczowie nie doszli, co najwięcej powiatowe piastując, ale
na polu Marsa[5] niepożyte krajowi oddali usługi, za które różnymi czasami
hojnie bywali nagradzani. Gniazdo ich rodzinne istniejące do dziś zwało się
także Billewicze, ale prócz nich posiadali wiele innych majętności i w okolicy
Rosień, i dalej ku Krakinowu[6], wedle Laudy, Szoi, Niewiaży[7] — aż hen,
jeszcze za Poniewieżem[8]. Potem rozpadli się na kilka domów, których członkowie
potracili się z oczu. Zjeżdżali się wszyscy wówczas tylko, gdy w Rosieniach, na
równinie zwanej Stany, odbywał się popis pospolitego ruszenia żmudzkiego.
Częściowo spotykali się także pod chorągwiami litewskiego komputu[9] i na
sejmikach, a że byli zamożni, wpływowi, więc liczyć się z nimi musieli sami
nawet wszechpotężni na Litwie i Żmudzi Radziwiłłowie.

\end{abstract}