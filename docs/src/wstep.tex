\chapter{Wstęp}

\begin{comment}
- motywacja, po co to robimy
- state of the art. Jak wyglada dziedzina, ktora sie zajmujemy?
Stosujemy tu metode szybkiego top down z ogolnikow do szczegolow.
Przechodzimy do tematyki, ktora sie zajmujemy. Piszemy co zrobiono precyzyjnie
w danej tematyce. Cytujemy artykuly (ksiazki - won!) Piszemy to w celu ..
- .. zdefiniowania celow pracy
- a potem jak te cele chcemy osiagnac, jakimi metodami
- potem piszemy co bedzie znajdowac sie w kazdym z rodzialow
- i na koncu jaki jest impact pracy
\end{comment}

W dzisiejszych czasach wpływ Internetu na życie codzienne jest niepodważalny.
Już od kilkunastu lat świat globalnej wioski przenika się z życiem realnym.
Nikogo nie dziwią prezentowane w kanałach informacyjnych komentarze
pochodzące z sieci, których autorami są zarówno osoby znane jak i zwykli
internauci. Rozrost Internetu przebiega w błyskawicznym tempie, a wydarzenia na
świecie komentowane są na żywo przez wielu ludzi. Aktualne trendy tworzone są na
blogach, mikroblogach czy serwisach społecznościowych.

Wyzwanie wobec ogromu tych informacji podejmuje dzisiejsza informatyka.
Przetwarzanie tak dużej ilości danych wymaga wielu zautomatyzowanych procesów.
W dzisiejszych czasach nie wystarczy już dowiedzieć się kto z kim najczęściej
się komunikuje, ale dużo bardziej interesujące jest to, o czym dany internauta
pisze i w jaki sposób to czyni.

Wielkie firmy chcą wiedzieć jak odbierane są ich produkty, jakie emocje
wzbudzają wśród klientów ich usługi i czy udaje im się spełniać ich oczekiwania.
Analiza użytkowników serwisów społecznościowych może być także bardzo
interesującym przedmiotem badań socjologów nad zmieniającym się społeczeństwem i
wpływem Internetu na ten proces.
Dodatkowo, analiza geolokalizacji może pozwolić marketingowcom na
odkrywanie nowych rejonów świata, w których mogliby oferować swoje
produkty i usługi.

Naprzeciw tym potrzebom budowane są systemy informatyczne, które potrafią takie
informacje uzyskać, przetwarzać i prezentować. Przykład takiego systemu został
zrealizowany w ramach tej\linebreak pracy~magisterskiej.

\section{Cel pracy}
Celem niniejszej pracy jest przedstawienie możliwości wykorzystania
analizy sentymentu oraz geolokacji w analizie użytkowników sieci
społecznościowych. Praca ta ma przedstawić czy możliwe połączenie tych trzech
dziedzin ze sobą i jeśli tak, to w jaki sposób to zrobić i co dzięki
nim nowego można się dowiedzieć.

Przedmiotem badań są użytkownicy serwisu mikroblogowego Twitter, a dokładniej
rzec biorąc osoby komentujące mecze piłkarskie w ramach angielskiej Premier
League.
Staram się odpowiedzieć na~pytania:
\begin{itemize}
  \item jak internauci korzystają z mediów społecznościowych,
  \item kiedy są najaktywniejsi,
  \item jakie wyrażają emocję,
  \item z jakich miejsc komentują,
  \item czy i w jakie grupy się łączą.
\end{itemize}

\section{Zawartość pracy}
W niniejszy dokumencie można wyróżnić 3 części. W pierwszej prezentuję
aktualny przegląd badań dotyczący poruszanych przeze mnie tematów, który
opisany został w rozdziale \ref{chapter:przegladbadan}. Można tam znaleźć
informacje na temat sieci społecznych, analizy sentymentu, geolokacji
a także o Twitterze w kontekście aktualnego stanu wiedzy na ich temat i sposobu
wykorzystania ich w nauce.

Przez drugą część można rozumieć rozdziały \ref{chapter:koncepcjarozwiazania}
oraz \ref{chapter:architektura}, gdzie prezentuję odpowiednio koncepcję
rozwiązania oraz architekturę systemu. W pierwszym rozdziale
przedstawiam koncepcję, metody i algorytmy, które zastosowałem w procesie
przeprowadzania badań. Omawiam tam również sposób w jaki dane były zbierane
i przetwarzane. Opisany jest algorytm badania wydźwięku wypowiedzi a także
sposoby badania sieci społecznych i geolokacji.

Trzecia część pracy to rozdział \ref{chapter:eksperymenty}, w którym
prezentowane są eksperymenty przeprowadzone na zebranych danych.
Koncentruję się na pokazaniu zastosowania Twittera w analizie
sentymentu, sieci społecznych i geolokalizacji starając się połączyć
ze sobą te trzy dziedziny.

%Każdy eksperyment jest opisany, posiada w miarę potrzeb załączony wykres,
%oraz wnioski, które można na jego podstawie wysnuć.
