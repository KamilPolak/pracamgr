\chapter{Wstęp}
\label{sec:wstęp}
W dzisiejszych czasach wpływ Internetu na życie codzienne jest niepodważalny.
Już od kilku lat świat globalnej wioski przenika się z życiem realnym.
Nikogo nie dziwią już prezentowane w kanałach informacyjnych komentarze
pochodzące z Internetu, których autorami są zarówno osoby znane jak i zwykli
internauci.  Rozrost sieci przebiega w błyskawicznym tempie.
Wydarzenia na świecie komentowane są na żywo przez wielu ludzi -- bez względu na
wiek, płeć czy zawód. Aktualne trendy tworzone są na blogach, mikroblogach czy
serwisach społecznościowych.

Wyzwanie wobec ogromu tych informacji podejmuje dzisiejsza informatyka.
Przetwarzanie tak dużej ilości danych wymaga wielu zautomatyzowanych procesów.
W dzisiejszych czasach nie wystarczy już dowiedzieć się kto z kim najczęściej
się komunikuje, ale dużo bardziej interesujące jest to, o czym dany internauta
pisze i w jaki sposób to czyni.

Wielkie firmy chcą wiedzieć jak odbierane są ich produkty, jakie emocje
wzbudzają wśród klientów ich usługi i czy udaje im się ich zadowolić.
Analiza użytkowników serwisów społecznościowych może być także bardzo
interesującym przedmiotem badań socjologów nad zmieniającym się społeczeństwem i
wpływem Internetu na ten proces.
Dodatkowa analiza geolokalizacji może pozwolić marketingowcom różnych firm na
odkrywanie nowych rejonów świata, w których ich firmy mogłyby oferować swoje
produkty i usługi.

Naprzeciw tym potrzebom budowane są systemy informatyczne, które potrafią takie
informacje uzyskać, przetwarzać i prezentować. Przykład takiego systemu został
zrealizowany w ramach tej\linebreak pracy~magisterskiej.

\section{Cel pracy}
Niniejsza praca skupia się na analizie zachowań użytkowników w wybranych
portalach społecznościowych. Przedmiotem badań są użytkownicy serwisu
mikroblogowego Twitter. W ramach pracy staram się
odpowiedzieć na pytania:
\begin{itemize}
  \item jak internauci korzystają z mediów społecznościowych,
  \item kiedy są najaktywniejsi,
  \item jakie wyrażają emocję,
  \item z jakich miejsc komentują,
  \item czy i w jakie grupy się łączą.
\end{itemize}

Analiza serwisów społecznościowych niesie ze sobą wiele wyzwań. Jako główne można wymienić:
\begin{itemize}
  \item przetwarzanie języka naturalnego -- wiele skrótów, wyrażeń slangowych,
błędów ortograficznych czy typograficznych, sklejanie wyrazów, używanie słów
zapożyczonych z obcych języków, itp.,
  \item ogromna ilość przetwarzanych informacji,
  \item duża liczba krótkich wiadomości,
  \item duża liczba danych zaszumionych -- wpisy reklamowe (SPAM), automatycznie 
wklejanie linków do blogów, innych serwisów społecznościowych, itp.
\end{itemize}

W związku z powyższym zebrane dane muszą być odpowiednio przetworzone i przefiltrowane
zanim zostaną przeprowadzone na nich jakiekolwiek operacje.

W ramach tej pracy pobrałem z serwisu Twitter w ciągu 3 miesięcy blisko 8 milionów
wpisów (w~tym także tych zawierających informacje o geolokalizacji), 
opracowałem metodą analizy sentymentu~-- czyli wydźwięku wypowiedzi (pozytywna,
negatywna lub neutralna) oraz stworzyłem narzędzie wspomagające
analizę zebranych danych -- wyświetlanie szerokiej gamy wykresów, prezentowanie
wpisów na mapie, informowanie o sentymencie -- z możliwością szerokiej selekcji danych do analizy.
