\chapter{Wstęp}

\begin{comment}
- motywacja, po co to robimy
- state of the art. Jak wyglada dziedzina, ktora sie zajmujemy?
Stosujemy tu metode szybkiego top down z ogolnikow do szczegolow.
Przechodzimy do tematyki, ktora sie zajmujemy. Piszemy co zrobiono precyzyjnie
w danej tematyce. Cytujemy artykuly (ksiazki - won!) Piszemy to w celu ..
- .. zdefiniowania celow pracy
- a potem jak te cele chcemy osiagnac, jakimi metodami
- potem piszemy co bedzie znajdowac sie w kazdym z rodzialow
- i na koncu jaki jest impact pracy
\end{comment}

W dzisiejszych czasach wpływ Internetu na życie codzienne jest niepodważalny.
Już od kilkunastu lat świat globalnej wioski przenika się z życiem realnym.
Nikogo nie dziwią prezentowane w kanałach informacyjnych komentarze
pochodzące z sieci, których autorami są zarówno osoby znane jak i zwykli
internauci. Rozrost Internetu przebiega w błyskawicznym tempie, a wydarzenia na
świecie komentowane są na żywo przez wielu ludzi. Aktualne trendy tworzone są na
blogach, mikroblogach czy serwisach społecznościowych.

Wyzwanie wobec ogromu tych informacji podejmuje dzisiejsza informatyka.
Przetwarzanie tak dużej ilości danych wymaga wielu zautomatyzowanych procesów.
Nie wystarczy już dowiedzieć się kto z kim najczęściej
się komunikuje, ale bardziej interesujące jest to, o czym dany internauta
pisze i w jaki sposób to czyni.

Wielkie firmy chcą wiedzieć jak odbierane są ich produkty, jakie emocje
wzbudzają wśród klientów ich usługi i czy udaje im się spełniać ich oczekiwania.
Analiza użytkowników serwisów społecznościowych może być także bardzo
interesującym przedmiotem badań socjologów nad zmieniającym się społeczeństwem i
wpływem Internetu na ten proces. Analiza geolokalizacji może pozwolić
marketingowcom na odkrywanie nowych rejonów świata, w których mogliby oferować swoje
produkty i usługi.

Naprzeciw tym potrzebom budowane są systemy informatyczne, które potrafią takie
informacje uzyskać, przetwarzać i prezentować. Przykład takiego systemu został
zrealizowany w ramach tej\linebreak pracy~magisterskiej.

\section{Cel pracy}
Celem niniejszej pracy jest przedstawienie możliwości wykorzystania
analizy sentymentu oraz geolokacji w analizie zachowań użytkowników sieci
społecznościowych. Praca ta ma przedstawić czy możliwe jest połączenie tych
trzech dziedzin ze sobą i jeśli tak, to w jaki sposób to zrobić i co dzięki
temu można się dowiedzieć. 
Analiza sentymentu polega na badania wydźwięku
wypowiedzi. Polega ona na określaniu danego tekstu jako pozytywnego, negatywnego lub neutralnego.
Zastosowanie komputerowych technik do badania sentymentu pozwala na dużo szybszą
ocenę pojedynczego tekstu lub grupy tekstów w porównaniu do ręcznego badania.
Geolokacja to technika identyfikacji geograficznego położenia osoby lub
urządzenia za pomocą cyfrowych danych przetwarzanych przy pomocy Internetu.
Poprzez zastosowanie nadajników GPS w nowoczesnych urządzeniach elektronicznych
(telefony, tablety) ich użytkownicy mogą udostępniać innym informacje o swojej
lokalizacji. Zastosowanie geolokacji pozwala odkrywać charakterystyczne
zachowania w różnych rejonach świata.
Analiza sieci społecznych zajmuje się badaniem struktury społecznej tworzonej
przez jednostki (osoby lub organizacji) i połączeniami między nimi. Spośród
trzech wymienionych dziedzin jest najdłużej badaną przez naukowców. Pomaga
odkrywać rodzaje połączeń między jednostkami, kierunki ewolucji danych grup oraz
przewidywać zmiany w nich zachodzące.

Przedmiotem niniejszych badań są użytkownicy serwisu mikroblogowego
Twitter\footnote{www.twitter.com}. Skupiono się na osobach komentujących mecze
piłkarskie angielskiej Premier League. Praca podejmuje następujące tematy:
\begin{itemize}
  \item jak internauci korzystają z mediów społecznościowych -- jakie
  charakterystyczne zachowania można zauważyć,
  \item kiedy są najaktywniejsi i tworzą najwięcej wpisów i komentarzy,
  \item jakie wyrażają emocję -- kiedy i dlaczego tworzą wpisy pozytywne, a
  kiedy negatywne i jaki ma to wpływ na innych,
  \item z jakich miejsc komentują i dlaczego, jak daleko znajdują się od
  siebie, jak to wpływa na interakcje między nimi,
  \item czy i jakie grupy tworzą, jak te grupy się zmieniają w czasie i
  dlaczego?
\end{itemize}

Do realizacji postawionych celów zbudowany został system
zbierający i przetwarzający dane z serwisu Twitter. Przeprowadzonych zostało
wiele eksperymentów mających odpowiedzieć na zadane pytania. Zaprezentowane
zostały ich wyniki i wyciągnięte wnioski.

\section{Struktura i zawartość pracy}
Pracę można podzielić trzy części. W pierwszej zaprezentowany jest
aktualny przegląd badań dotyczący poruszanych tematów, który
opisany został w rozdziale \ref{chapter:przegladbadan}. Są to 
informacje na temat sieci społecznych, analizy sentymentu, geolokacji
a także dotyczące Twittera w kontekście aktualnego stanu wiedzy na ich temat i
sposobu wykorzystania ich w nauce.

Drugą część stanowią rozdziały \ref{chapter:koncepcjarozwiazania} oraz
\ref{chapter:architektura}, gdzie zaprezentowano koncepcję
rozwiązania oraz architekturę systemu. W pierwszym z nich przedstawiona jest
koncepcja, metody i algorytmy, które zostały zastosowane w procesie
przeprowadzania badań. Omówiony tam jest sposób w jaki dane były
zbierane i przetwarzane. Opisany jest także algorytm badania wydźwięku
wypowiedzi oraz sposoby badania sieci społecznych i geolokacji. W drugim zaś
przedstawiona jest zastosowana architektura systemu i wykorzystane narzędzia.

Trzecia część pracy to rozdział \ref{chapter:eksperymenty}, w którym
zaprezentowane są przeprowadzone eksperymenty. Skupiają się one na pokazaniu
zastosowania Twittera w analizie sentymentu, sieci społecznych i geolokalizacji.
Nacisk położono na to by pokazać jak połączyć ze sobą te trzy dziedziny.

%Każdy eksperyment jest opisany, posiada w miarę potrzeb załączony wykres,
%oraz wnioski, które można na jego podstawie wysnuć.
