\begin{abstract}

Analiza sieci społecznych to nauka o kilkudziesięcioletniej historii. Skupia się
na badaniu połączeń między użytkownikami, odkrywaniu grup między nimi, 
a także rodzajach relacji jakie ~\mbox{między sobą tworzą}.
\begin{comment}
Ogromny rozwój Internetu w XXI wieku doprowadził do zupełnie nowych możliwości
badania sieci społecznych. Rozwój serwisów społecznościowych dał ludzkości
dostęp do danych do tej pory nieosiągalnych. W tym momencie możliwe jest
nie tylko zaglądnięcie do tego jakie osoby tworzą sieć społeczną,
ale także do tego jakie komunikaty między sobą przekazują.
Daje to ogromne możliwości związane z badaniem treści tych komunikatów
i powiązaniem tego z sieciami społecznymi.
\end{comment}

W niniejszej pracy staram się pokazać w jaki sposób można połączyć ze sobą
analizę sentymentu (badanie wydźwięku -- określanie wypowiedzi jako pozytywne
lub negatywne) i geolokalizację w celu bogatszego badania sieci społecznych. Ich
zastosowanie otwiera możliwość odkrycia ciągów przyczynowo-skutkowych w procesie
tworzenia się sieci społecznych.

W tym celu zebrane zostało blisko 8 milionów wpisów z serwisu Twitter,
których autorami było ponad 1,5 miliona internautów. 
Zebrane dane były ukierunkowane na środowisko
piłkarskie związane z najwyższą klasą rozgrywkową w Anglii i Walii -- 
Barclays Premier League w sezonie 2013/2014.

Praca zawiera opis zastosowanych technik i przykłady ich wykorzystania na 
zebranych danych. Udowadnia, że przy ich pomocy można wysoce zautomatyzować
sposób badania nastrojów społecznych z uwzględnieniem lokalizacji geograficznej
czy grup, które wspólnie tworzą. Może być zastosowana jako 
\textit{proof-of-concept} (z ang. potwierdzenie, dowód koncepcji)
w badaniach na szerszą skalę niż tu zaprezentowana.  





\end{abstract}