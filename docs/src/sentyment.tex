\section{Badanie sentymentu}
\label{sec:sentyment}
\subsection{Narzędzie: TextBlob - Simplified Text Processing}
\texttt{TextBlob} (\url{https://textblob.readthedocs.org}) jest biblioteką dla
Pythona (2 i 3) stworzoną do przetwarzania danych tekstowych.
Udostępnia proste API dla popularnych operacji związanych z przetwarzaniem
języka naturalnego, takich jak klasyfikacja części mowy, ekstrakcja podmiotów
zdań, \textbf{analiza sentymentu, klasyfikacja}, tłumaczenie i wiele innych.

TextBlob jest zbudowany z wykorzystaniem innych narzędzi związanych z
przetwarzaniem tekstu: \texttt{NLTK} (\url{http://nltk.org/}) i \texttt{pattern}
(\url{http://www.clips.ua.ac.be/pages/pattern-en}).

\subsubsection{Klasyfikacja tekstu}
\texttt{TextBlob} dostarcza narzędzie do klasyfikacji tekstu na podstawie zbioru
testowego, wykorzystując do tego naiwnego klasyfikatora bayesowskiego.
Posiadając więc taki zbiór testowy, możliwe jest sklasyfikowanie tekstu na
przykład pod kątem sentymentu. Możliwa jest ocena całego tekstu lub z podziałem
na zdania, ocena prawdopodobieństwa każdej z klas czy sprawdzenie dokładności
klasyfikacji zbioru testowego.

\subsubsection{Analiza sentymentu}
Wyżej wymieniona biblioteka zawiera także dwie implementacje analizy sentymentu.
Są to \texttt{PatternAnalyzer} (domyślna - pochodząca z biblioteki
\texttt{pattern}) oraz \texttt{NaiveBayesAnalyzer} (\texttt{NTLK}).

\texttt{PatternAnalyzer} korzysta z leksykonu przymiotników angielskich
ocenionych pod kątem ich sentymentu w skali od -1.0 do +1.0 (-1.0 -- negatywny,
+1.0 - pozytywny).
Klasyfikator zwraca średnią arytmetyczną spośród wszystkich przymiotników, które
pojawiły się w tekście - czyli im wartość bliżej +1.0, tym sentyment bardziej
pozytywny, a im bliżej -1.0, tym sentyment bardziej negatywny.

\texttt{NaiveBayesAnalyzer} został wytrenowany na korpusie z recenzjami filmów
(ocenionych jako negatywne lub pozytywne), zwraca prawdopodobieństwa, że
sentyment jest pozytywny lub negatywny oraz klasyfikuje jednoznacznie na
podstawie tego, która z tych wartości jest wyższa.
